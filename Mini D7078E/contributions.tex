\section{Contributions}

Both group members of Group 35 contributed actively to the successful completion of this D7078E Cloud Security Mini Project assignment. \textbf{Member 1} was responsible for the infrastructure design and deployment across Tasks 1-3. This included designing and implementing the Python-based web application with three endpoints (/, /metrics, /burn), configuring the EC2 instance with Ubuntu 22.04 LTS, creating the AMI template, architecting the complete infrastructure stack with Application Load Balancer across multiple availability zones, configuring the Auto Scaling Group with min=1, desired=1, max=3, implementing both scale-out and scale-in policies using AWS CLI with precise thresholds (80\% for scale-out, 30\% for scale-in), creating two CloudWatch alarms with correct evaluation periods, and designing a comprehensive monitoring dashboard with real-time widgets for CPU Utilization, RequestCount, HealthyHostCount, and UnhealthyHostCount.

\textbf{Member 2} was responsible for load testing, failure simulation, and performance validation across Tasks 4-5. This included developing the Python asyncio-based load generator (agent.py) with configurable workers and RPS parameters, creating the Dockerfile and docker-compose.yml for containerized deployment, executing comprehensive three-phase load testing (baseline, medium, and high load) while continuously monitoring CloudWatch metrics, orchestrating controlled failure simulation using AWS Systems Manager with 90\% CPU stress for 120 seconds, documenting the complete failure detection and recovery timeline (30 seconds to detection, 2-3 minutes total recovery), and capturing critical screenshots at each milestone documenting the scaling progression from 1 to 3 instances and automatic recovery. Both members collaborated on Task 6 to systematically delete all AWS resources in the correct order, verify complete cleanup, and produce the comprehensive final report with detailed analysis, performance metrics, and lessons learned.
